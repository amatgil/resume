%%%%%%%%%%%%%%%%%
% From altacv

%% Use the "normalphoto" option if you want a normal photo instead of cropped to a circle
% \documentclass[10pt,a4paper,normalphoto]{altacv}
% \documentclass[10pt,a4paper,normalphoto]{altacv}

\documentclass[10pt,a4paper,ragged2e,withhyper]{altacv}
%% AltaCV uses the fontawesome5 and simpleicons packages.
%% See http://texdoc.net/pkg/fontawesome5 and http://texdoc.net/pkg/simpleicons for full list of symbols.

% Change the page layout if you need to
\geometry{left=1.25cm,right=1.25cm,top=1.5cm,bottom=1.5cm,columnsep=1.2cm}

% The paracol package lets you typeset columns of text in parallel
\usepackage{paracol}
\usepackage{tikz}
\usepackage{xcolor}
\usetikzlibrary{positioning}
\usepackage[colorlinks = true, % none of these do anything
            linkcolor = blue,
            urlcolor  = blue,
            citecolor = blue,
            anchorcolor = blue]{hyperref}

\definecolor{caslinkblue}{RGB}{0,102,204}  % IE color link ftw
\newcommand{\casurl}[1]{{\color{caslinkblue}\url{#1}}}
% WHEN COMPILING WITH XELATEX PLEASE USE
% xelatex -shell-escape -output-driver="xdvipdfmx -z 0" sample.tex
\iftutex 
  % If using xelatex or lualatex:
  \setmainfont{Roboto Slab}
  \setsansfont{Lato}
  \renewcommand{\familydefault}{\sfdefault}
\else
  % If using pdflatex:
  \usepackage[rm]{roboto}
  \usepackage[defaultsans]{lato}
  % \usepackage{sourcesanspro}
  \renewcommand{\familydefault}{\sfdefault}
\fi

% Change the colours if you want to
\definecolor{SlateGrey}{HTML}{2E2E2E}
\definecolor{LightGrey}{HTML}{666666}
\definecolor{DarkPastelRed}{HTML}{450808}
\definecolor{PastelRed}{HTML}{8F0D0D}
\definecolor{GoldenEarth}{HTML}{E7D192}
\colorlet{name}{black}
\colorlet{tagline}{PastelRed}
\colorlet{heading}{DarkPastelRed}
\colorlet{headingrule}{GoldenEarth}
\colorlet{subheading}{PastelRed}
\colorlet{accent}{PastelRed}
\colorlet{emphasis}{SlateGrey}
\colorlet{body}{LightGrey}

% Change some fonts, if necessary
\renewcommand{\namefont}{\Huge\rmfamily\bfseries}
\renewcommand{\personalinfofont}{\footnotesize}
\renewcommand{\cvsectionfont}{\LARGE\rmfamily\bfseries}
\renewcommand{\cvsubsectionfont}{\large\bfseries}


% Change the bullets for itemize and rating marker
% for \cvskill if you want to
\renewcommand{\cvItemMarker}{{\small\textbullet}}
\renewcommand{\cvRatingMarker}{\faCircle}
% ...and the markers for the date/location for \cvevent
% \renewcommand{\cvDateMarker}{\faCalendar*[regular]}
% \renewcommand{\cvLocationMarker}{\faMapMarker*}


% If your CV/résumé is in a language other than English,
% then you probably want to change these so that when you
% copy-paste from the PDF or run pdftotext, the location
% and date marker icons for \cvevent will paste as correct
% translations. For example Spanish:
% \renewcommand{\locationname}{Ubicación}
% \renewcommand{\datename}{Fecha}


%% Use (and optionally edit if necessary) this .tex if you
%% want to use an author-year reference style like APA(6)
%% for your publication list
% \input{pubs-authoryear.tex}

%% Use (and optionally edit if necessary) this .tex if you
%% want an originally numerical reference style like IEEE
%% for your publication list
\input{pubs-num.tex}

\begin{document}
\name{Amat Gil Viñes}
\tagline{Student}
%% You can add multiple photos on the left or right
%\photoR{2.8cm}{Globe_High}
% \photoL{2.5cm}{Yacht_High,Suitcase_High}

\personalinfo{%
  % Not all of these are required!
  \email{amatgilvinyes@gmail.com}
  %\phone{000-00-0000}
  \location{Barcelona, Barcelona}
  \homepage{www.amatgil.cat}
  % \twitter{@twitterhandle}
  \github{www.github.com/amatgil}
  %% You can add your own arbitrary detail with
  %% \printinfo{symbol}{detail}[optional hyperlink prefix]
  % \printinfo{\faPaw}{Hey ho!}[https://example.com/]

  %% Or you can declare your own field with
  %% \NewInfoFiled{fieldname}{symbol}[optional hyperlink prefix] and use it:
  % \NewInfoField{gitlab}{\faGitlab}[https://gitlab.com/]
  % \gitlab{your_id}
  %%
  %% For services and platforms like Mastodon where there isn't a
  %% straightforward relation between the user ID/nickname and the hyperlink,
  %% you can use \printinfo directly e.g.
  % \printinfo{\faMastodon}{@username@instace}[https://instance.url/@username]
  %% But if you absolutely want to create new dedicated info fields for
  %% such platforms, then use \NewInfoField* with a star:
  % \NewInfoField*{mastodon}{\faMastodon}
  %% then you can use \mastodon, with TWO arguments where the 2nd argument is
  %% the full hyperlink.
  % \mastodon{@username@instance}{https://instance.url/@username}
}

\makecvheader
%% Depending on your tastes, you may want to make fonts of itemize environments slightly smaller
% \AtBeginEnvironment{itemize}{\small}

%% Set the left/right column width ratio to 6:4.
\columnratio{0.6}

% Start a 2-column paracol. Both the left and right columns will automatically
% break across pages if things get too long.
\begin{paracol}{2}
\cvsection{Projects}

\cvevent{S-IC-mulador}{A fully fledged (custom) assembly emulator, featuring memory, memory-mapped IO and and execution analysis}{}{}
\begin{itemize}
\item Interactive link: \casurl{https://amatgil.cat/utils/s-ic-mulador}
\item Language: Rust
\end{itemize}

\divider

\cvevent{Tubaitu}{Hand-made graphical 2x2 Rubix Cube featuring standard half-moves, scrambling and a home-brewn solving algorithm}{}{}
\begin{itemize}
\item Interactive link: \casurl{https://amatgil.cat/jocs_misc/cub_tubaitu}
\item Language: Rust
\end{itemize}
\divider 

\cvevent{Fractalitzador}{Capable of drawing concrete fractals as well as from their L-System grammar specification (with adjustable parameters)}{}{}
\begin{itemize}
\item Interactive link: \casurl{https://amatgil.cat/jocs_misc/fractals}
\item Language: Rust
\end{itemize}
\divider 

\cvevent{Tau Memorizer}{A gamified applet to memorize, digit by digit, the value of the true circle constant: tau}{}{}
\begin{itemize}
\item Interactive link: \casurl{https://amatgil.cat/jocs_misc/tau_mem}
\item Language: JavaScript (+ HTML + CSS)
\end{itemize}
\divider 

\cvevent{Game of Life}{A performant, scalable, interactive Game of Life simulator with a strong focus on UX}{}{}
\begin{itemize}
\item Interactive link: \casurl{https://amatgil.cat/altres/game_of_life}
\item Language: Rust
\end{itemize}
\divider 

\cvevent{Boron}{An expression-driven scripting language based exclusively around prefix expressions}{}{}
\begin{itemize}
\item Source code: \casurl{https://github.com/amatgil/boron/tree/master/examples}
\item Language: Haskell
\end{itemize}

\divider

\switchcolumn
\bigskip
\bigskip

\divider
\cvevent{Resume/CV}{The document you're looking at!}{}{}
\begin{itemize}
\item Source code: \casurl{https://github.com/amatgil/resume}
\item Language: LaTeX
\end{itemize}
\divider

\cvevent{Wawa}{One of a kind, uiua-specialized discord bot}{}{}
\begin{itemize}
\item Source code: \casurl{https://github.com/amatgil/wawa}
\item Language: Rust
\end{itemize}
\divider

\cvevent{Breakbreak}{Breakout, written as to maximize code readability, ergonomics and user experience (UX), in a concatenative array language}{}{}
\begin{itemize}
\item Source code: \casurl{https://github.com/amatgil/breakbreak}
\item Language: Uiua
\end{itemize}


\medskip
\cvsection{Languages}

\cvtag{Rust}
\cvtag{Uiua}
\cvtag{Haskell}\\
\cvtag{Common Lisp}
\cvtag{LaTeX}\\
%\begin{tikzpicture}
%  \node[inner sep=0pt] (rustlang) {
%    \includegraphics[width=50px]{rustlogo.png}
%    Rust
%  };
%  \node[inner sep=0pt] (uiualang) [right=of rustlang] {
%    \includegraphics[width=50px]{Uiua-logo.png}
%  };
%  \node[inner sep=0pt] (haskelllang) [below=of rustlang] {
%    \includegraphics[width=50px]{haskell-logo.png}
%  };
%  \node[inner sep=0pt] (commonlisplang) [right=of haskelllang]  {
%    \includegraphics[width=50px]{Lisp_logo.png}
%  };
%  \node[inner sep=0pt] (latexlang) [below=of haskelllang] {
%    \includegraphics[width=50px]{LaTeX_logo.svg.png}
%  };
%\end{tikzpicture}

%\cvsection{My Life Philosophy}
%
%\begin{quote}
%``''
%\end{quote}

%\cvsection{Strengths}
%
%% Don't overuse these \cvtag boxes — they're just eye-candies and not essential. If something doesn't fit on a single line, it probably works better as part of an itemized list (probably inlined itemized list), or just as a comma-separated list of strengths.
%
%% The `ragged2e` document class option might cause automatic linebreaks between \cvtag to fail.
%% Either remove the ragged2e option; or 
%% add \LaTeXraggedright in the paragraph for these \cvtag
%{\Xraggedright
%\cvtag{Hard-working}
%\cvtag{Curious}
%\cvtag{Motivator \& Leader}
%\par}
%
%\divider\smallskip
%
%%% ...Or manually add linebreaks yourself
%\cvtag{C++}
%\cvtag{Embedded Systems}\\
%\cvtag{Statistical Analysis}

\medskip
\cvsection{Paradigms}
%\cvachievement{\includegraphics[width=22px]{rustlogo.png}}{Rust}{Strongly typed, imperative paradigm}
%
%\divider
%
%\cvachievement{\includegraphics[width=21px]{Uiua-logo.png}}{Uiua}{Dynamically typed, array \& stack}
%
%\divider
%
%\cvachievement{\includegraphics[width=21px]{haskell-logo.png}}{Haskell}{Strongly typed, pure functional}
%
%\divider
%
%\cvachievement{\includegraphics[width=21px]{Lisp_logo.png}}{Common Lisp}{Dynamically typed, lisp-y}
%
%\cvsection{Spoken Languages}
\cvtag{Strong typing}
\cvtag{Dynamic typing}\\
\cvtag{Lisp (meta)}
\cvtag{Array \& stack based}\\
\cvtag{Functional}
\cvtag{Imperative}\\
%\begin{tikzpicture}
%  \node[inner sep=0pt] (strong) {
%    Strong typing   
%  };
%  \node[inner sep=0pt] (dyn) [right = 1.0cm of strong] {
%    Dynamic typing
%  };
%  \node[inner sep=0pt] (array) [below = 0.2cm of strong]  {
%    Array \& stack-based
%  };
%  \node[inner sep=0pt] (imperative) [below = 0.2cm of array] {
%    Imperative
%  };
%  \node[inner sep=0pt] (functional) [below= 0.2cm of dyn] {
%    Functional, lisp
%  };
%\end{tikzpicture}

\medskip
\cvsection{Spoken Languages}
\cvachievement{}{English}{Certified Proficiency, C2}
\divider

\cvachievement{}{Spanish}{Native}
\divider

\cvachievement{}{Catalan}{Native}
\divider

%\cvskill{German}{0.5} %% Supports X.5 values.

%% Yeah I didn't spend too much time making all the
%% spacing consistent... sorry. Use \smallskip, \medskip,
%% \bigskip, \vspace etc to make adjustments.
%\medskip

%\cvsection{Education}
%
%\cvevent{Ph.D.\ in Your Discipline}{Your University}{Sept 2002 -- June 2006}{}
%Thesis title: Wonderful Research
%
%\divider
%
%\cvevent{M.Sc.\ in Your Discipline}{Your University}{Sept 2001 -- June 2002}{}
%
%\divider
%
%\cvevent{B.Sc.\ in Your Discipline}{Stanford University}{Sept 1998 -- June 2001}{}

% \divider

%\cvsection{Referees}
%
%% \cvref{name}{email}{mailing address}
%\cvref{Prof.\ Alpha Beta}{Institute}{a.beta@university.edu}
%{Address Line 1\\Address line 2}
%
%\divider
%
%\cvref{Prof.\ Gamma Delta}{Institute}{g.delta@university.edu}
%{Address Line 1\\Address line 2}


\end{paracol}


\end{document}
